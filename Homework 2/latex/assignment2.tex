\documentclass{article}

\usepackage{tikz-qtree}
\usepackage{tikz}

\author{Yeoun Chan Kim \and John Strauser \and Xuanang Wang}

\title{CS440 Assignment2}

\begin{document}

\maketitle

\section*{Part 9}

\hspace{5mm} 

a) Hill climbing works better than simulated annealing when the objective function has only one global maximum and without plateaux.

b) The hill-climbing part of simulated annealing is not necessary when the maximum (peek) value in the objective function is relatively or completely flat.

c) Simulated annealing is useful when the local/global maximum of objective function is relatively sharp, the objective function has a lot local maximum with plateaux or shoulder situation.

d)

\section*{Part 10}

\tikzset{every tree node/.style={minimum width=2em,draw,blank},
         blank/.style={draw=none},
         edge from parent/.style=
         {draw,edge from parent path={(\tikzparentnode) -- (\tikzchildnode)}},
         level distance=1.5cm}
\begin{tikzpicture}
\Tree
[.(1,4)
\edge[blank]; \node[blank]{};
\edge[];
    [.(2,4)
    \edge[blank]; \node[blank]{};
    \edge[];
        [.(2,3)
        \edge[];
            [.(1,3)
                \edge[];
                [.(1,2)
                \edge[];
                    [.(3,2)
                    \edge[];
                        [.\framebox{(1,3)}
                            \edge[blank];{-1}
                        ]
                    ]
                ]
                \edge[];
                    [.\framebox{\framebox{(1,4)}}
                        \edge[blank];(?)
                    ]
            ]
            \edge[];
                [.(\framebox{4,3})
                    \edge[blank];{1}
                ]
        ]
    \edge[blank]; \node[blank]{};
    ]
\edge[blank]; \node[blank]{};
]
\end{tikzpicture}
\end{document}
